\documentclass[UTF8]{ctexart}

\begin{document}
\section{CCD相机特性简介}
\subsection{CCD传感器原理简述}
电荷耦合器件的突出特点是以电荷作为信号,而不同于其他大多数传感器以电流或者电压为信号。CCD的基本功能是完成电荷的存储和电荷的转移。CCD图像传感器的基本工作原理包括: 
 信号电荷的产生与存储 
 电荷的传输转移 
 电荷的检测输出 

\section{配套软件} 
\subsection*{前言}
作为具有物理背景又同时具有软件开发能力的团队,我们在平时的学习中会接触很多的实验配套软件。
诚然,这些软件或多或少的都确实能够辅助实验的操作,帮助学生达成实验目的,但是,
在真正的使用过程中,几乎总是会遇到各种各样的问题,如界面晦涩难懂,需要安装兼容性很差的老旧驱动,不适配新操作系统等。
在之前的工作中,我们使用C++ Qt框架结合了相机厂商的专用SDK进行了软件开发,虽然性能的确不错,但是C++语言存在着抽象层次较低,没有现存的异步IO框架等限制。
最重要的一点是,这样开发出来的程序只能适配某个厂商的设备,而对不同的厂商设备完全没有兼容能力。另一方面,如果要实现跨平台,C++对应的不同构建工具的工作量是非常庞大的。
上述限制与我们的构想不符。

Web技术是现今软件技术发展中的重要组成部分之一。而今,由于浏览器和脚本引擎的渐趋成熟,尽管仍有争吵的声音,但不可否认的是Web技术已经出现了向桌面应用开发渗透的趋势。
Electron就是其中的优秀代表。Electron依托于内建的Chromium和Node.js,利用进程通信实现了利用Web技术进行界面显示的同时支持了原生API的调用,在构建有一定复杂度且交互较多的界面时
提供了无与伦比的灵活性,并且利用上下文隔离提高了Web前端的安全性;而谷歌在chromium集成的WebRTC作为一个丰富而强大的实时音视频通话API,通过在windows下调用DirectShow,在linux下调用V4L2,在macOS下调用
AVFundation实现了跨平台的相机驱动支持,并由于C++的天性,拥有着优越的性能。

我们经过考虑,决定将软件的技术栈迁移至Electron+vue3+typescript,在vue的基础之上利用element-plus框架快速实现界面,
并使用node.js的跨平台原生模块实现OpenCV的调用,从而实现在需要的地方不对性能做出任何妥协的前提下,以最低的开发成本实现了友好的界面,以及目前技术条件下最广泛的兼容性。
\subsection{简介}
我们最终将系统的配套软件命名为quantum-element。quantum-element是基于Electron和vue的跨平台桌面应用程序,并具有低成本移植到移动设备的潜力。我们通过谷歌在Chromium中集成的WebRTC API实现
在现行的大部分操作系统下对实现了标准驱动的相机的访问与控制(并不是所有的相机都完全实现了标准的驱动,某些厂商的相机不能通过WebRTC API调整曝光等参数),使得系统整体的搭建成本可以藉由使用廉价的摄像头等
方式进一步地降低;
\end{document}